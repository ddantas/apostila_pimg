\chapter{\emph{Image Enhancement} (melhoramento) no domínio da frequência}

Capítulo 4 de Gonzalez, \textit{Digital Image Processing}~\cite{gonzalez2006image}.

%%%%%%%%%%%%%%%%%%%%%%%%%%%%%%%%%%%%%%%%%%%%%%%%%%%%%%%%%%%%
\section{Introdução}

\begin{easylist}

  & Transformada de Fourier é uma maneira de representar um sinal como uma integral de senos e cossenos multiplicados por uma função de peso. Um sinal pode ser convertido para sua representação transformada e reconvertido de volta para o domínio original sem perda de informação. Também é possível realizar operações como filtragens na representação transformada e reconverter de volta para o domínio original.

\end{easylist}
  
%%%%%%%%%%%%%%%%%%%%%%%%%%%%%%%%%%%%%%%%%%%%%%%%%%%%%%%%%%%%
\section{A transformada de Fourier}

\begin{easylist}

  & Transformada de Fourier $g(u)$ de uma função contínua de uma única variável é dada pela fórmula

  \[ g(u) = \int^{\infty}_{-\infty} f(x) e^{-j2\pi ux} dx \]

  onde $j = \sqrt{-1}$, De maneira análoga, dado $g(u)$, podemos obter $f(x)$ através da transformada inversa de Fourier
  
  \[ f(x) = \int^{\infty}_{-\infty} g(u) e^{ j2\pi ux} du \]

\vspace{1cm}
  
  & Essas operações possuem uma versão para duas variáveis

  \[ g(u, v) = \int^{\infty}_{-\infty}\int^{\infty}_{-\infty} f(x, y) e^{-j2\pi (ux + vy)} dx dy \]

  e sua inversa

  \[ f(x, y) = \int^{\infty}_{-\infty}\int^{\infty}_{-\infty} g(u, v) e^{ j2\pi (ux + vy)} dx dy \]

  & Muitas vezes é mais fácil manipular as versões contínuas da transformada de Fourier, mas para trabalhar com imagens, usamos suas versões discretas

  \[ g(u) = \frac 1M \sum^{M-1}_{x=0} f(x) e^{-j2\pi ux/M} \textrm{para u de 0 a $M-1$} \]

  e sua inversa

  \[ f(x) =          \sum^{M-1}_{x=0} g(u) e^{ j2\pi ux/M} \textrm{para x de 0 a $M-1$} \]

  & O conceito de domínio da frequência segue a fórmula de Euler

  \[ e^{j\theta} = \cos\theta + j\sin\theta \]

  & Substituindo na transformada direta discreta, temos

  \[ g(u) = \frac 1M \sum^{M-1}_{x=0} f(x) (\cos (2\pi ux/M) -j\sin (2\pi ux/M)) \]

  & e na transformada inversa discreta, temos

  \[ g(u) = \sum^{M-1}_{x=0} f(x) (\cos (2\pi ux/M) +j\sin (2\pi ux/M)) \]

  & A transformada de Fourier discreta bidimensional pode ser calculada pela fórmula

  \[ g(u, v) = \frac 1{MN} \sum^{M-1}_{x=0}\sum^{N-1}_{y=0} f(x, y) e^{-j2\pi (ux/M + vy/N)} \].

  Nesse caso, a inversa pode ser calculada pela fórmula

  \[ f(x, y) =             \sum^{M-1}_{x=0}\sum^{N-1}_{y=0} g(u, v) e^{ j2\pi (ux/M + vy/N)} \].

\clearpage
  
  & Substituindo a fórmula de Euler na transformada discreta bidimensional de Fourier, obtemos

  \[ g(u, v) = \frac 1{MN} \sum^{M-1}_{x=0}\sum^{N-1}_{y=0} f(x, y)
      \cos\left(2\pi \left(\frac{ux}{M} + \frac{vy}{N}\right)\right) -
     j\sin\left(2\pi \left(\frac{ux}{M} + \frac{vy}{N}\right)\right)   \]

  e sua inversa

  \[ f(x, y) =             \sum^{M-1}_{x=0}\sum^{N-1}_{y=0} g(u, v)
      \cos\left(2\pi \left(\frac{ux}{M} + \frac{vy}{N}\right)\right) +
     j\sin\left(2\pi \left(\frac{ux}{M} + \frac{vy}{N}\right)\right)   \]

\end{easylist}
  
