\chapter{Processamento morfológico de imagens}

Capítulo 9 de Gonzalez, \textit{Digital Image Processing}~\cite{gonzalez2006image}.


%%%%%%%%%%%%%%%%%%%%%%%%%%%%%%%%%%%%%%%%%%%%%%%%%%%%%%%%%%%%
\section{Introdução}

\begin{easylist}

  & Morfologia Matemática denota um conjunto de ferramentas úteis para representar e descrever formas de regiões da imagem, como bordas, esqueletos e casco convexo. Também são úteis como operações de pré e pós-processamento, como filtragem morfológica e afinamento.
  
  & Duas operações fundamentais da Morfologia Matemática são erosão e dilatação.

\end{easylist}


%%%%%%%%%%%%%%%%%%%%%%%%%%%%%%%%%%%%%%%%%%%%%%%%%%%%%%%%%%%%
\section{Erosão e dilatação}

\begin{easylist}

  & Erosão: seja $f$ uma imagem digital em escala de cinza e $S$ um conjunto de pixels centrado na origem. Denotamos erosão por

  \[ (f \ominus S)(x) = \min\{ f(x+z) | z \in S \} \]

  onde $S$ é conhecido como elemento estruturante e $x$ são as coordenadas dos pixels de $f$.
  
  & Dilatação: seja $f$ uma imagem digital em escala de cinza e $S$ um conjunto de pixels centrado na origem. Denotamos dilatação por

  \[ (f \oplus S)(x) = \max\{ f(x-z) | z \in S \} \]

  onde $S$ é conhecido como elemento estruturante e $x$ são as coordenadas dos pixels de $f$.

  & Dualidade: a erosão e dilatação são duais um do outro com relação ao complemento e reflexão do elemento estruturante na medida em que

  \[ (f \ominus S)^c = f^c \oplus \hat{S} \]

ou 
  
  \[ (f \oplus S)^c = f^c \ominus \hat{S} \]

  onde o $c$ superescrito denota o complementar de imagens binárias ou a negativa de imagens em escala de cinza, e $\hat{S}$ denota a reflexão do elemento estruturante $S$ em relação à origem. Algumas implementações da dilatação e erosão podem não preservar a dualidade a depender de suas especificidades.

\end{easylist}


%%%%%%%%%%%%%%%%%%%%%%%%%%%%%%%%%%%%%%%%%%%%%%%%%%%%%%%%%%%%
\section{Abertura e fechamento}

\begin{easylist}

  & Abertura de $f$ pelo elemento estruturante $S$ é denotada por $f \circ S$ e é definida por

  \[ f \circ S = (f \ominus S) \oplus S. \]

  & Fechamento de $f$ pelo elemento estruturante $S$ é denotada por $f \bullet S$ e é definida por

  \[ f \bullet S = (f \oplus S) \ominus S. \]

\end{easylist}


%%%%%%%%%%%%%%%%%%%%%%%%%%%%%%%%%%%%%%%%%%%%%%%%%%%%%%%%%%%%
\section{Operador \textit{hit-miss}}

\begin{easylist}

  & O operador \textit{hit-miss} é um operador usado para detecção de formas (\textit{shape detection}). O elemento estruturante usado pode conter, além dos valores 0 e 1, o valor X, ou \textit{don't care}. O elemento estruturante

  \end{easylist}

  \[
    B = 
    \begin{bmatrix}
      X & 0 & 0 \\
      1 & 1 & 0 \\
      X & 1 & X \\
    \end{bmatrix}  
  \]
    
\begin{easylist}

  e suas rotações por múltiplos de $90^\circ$ pode ser usado para detectar cantos. Seu complemento, necessário no processo, é

\end{easylist}
  
  \[
    B = 
    \begin{bmatrix}
      X & 1 & 1 \\
      0 & 0 & 1 \\
      X & 0 & X \\
    \end{bmatrix}  
  \]

\begin{easylist}

  & O operador \textit{hit-miss} pelo elemento estruturante $B$ da imagem $f$ é denotado por $f \circledast B$ e é dado por

  \[ f \circledast B = (f \ominus B) \cap (f^c \ominus B^c). \]

  & Considere que $B_\alpha$ é o elemento estruturante $B$ rotacionado pelo ângulo $\alpha$ no sentido anti-horário. Poderíamos obter todos os cantos da imagem através da operação abaixo.

  \[ (f \circledast B_{0})  \cup
     (f \circledast B_{90}) \cup
     (f \circledast B_{180}) \cup
     (f \circledast B_{270})      \]

\end{easylist}


%%%%%%%%%%%%%%%%%%%%%%%%%%%%%%%%%%%%%%%%%%%%%%%%%%%%%%%%%%%%
\section{Algoritmos morfológicos básicos}

\begin{easylist}

  & Extração de borda: a borda de uma imagem f, denotada por $\beta(f)$, pode ser obtida por uma erosão de $f$ pelo elemento estruturante $B$, e depois fazendo a diferença entre $f$ e sua erosão, ou seja,

  \[ \beta(f) = f - (f \ominus B). \]

  São alternativas a esse operador  

  \[ \beta_d(f) = (f \oplus B) - f \]

  e

  \[ \beta_c(f) = (f \oplus B) - (f \ominus B). \]

  & Preenchimento de buracos: Seja $A$ um conjunto de pixels com valor 1 4-conectados ao redor de uma região de pixels com valor 0, ou seja, um buraco. Seja $p$ um pixel pertencente ao buraco. Nosso objetivo é preencher o buraco inteiro com pixels de valor 1. O procedimento a seguir preenche o buraco:

  \medskip
  \hspace{1cm}    \textit{do}
  
  \hspace{1cm}\hspace{1cm}    $X_k = (X_{k-1} \oplus B) \cap A^c, k = 1, 2, 3\dots$

  \hspace{1cm}    \textit{while} $X_k \neq X_{k-1}$
  \medskip

  onde $X_0 = p$ e $B$ é o quadrado $3\times3$. Ao final, a união de $X_k$ e $A$ nos dá a região preenchida.

  & Extração de componentes conexos: Seja $Y \subseteq A$ um componente conexo, e $p \in Y$ um pixel. O procedimento a seguir nos dá todos os pontos de $Y$:

  \medskip
  \hspace{1cm}    \textit{do}
  
  \hspace{1cm}\hspace{1cm}    $X_k = (X_{k-1} \oplus B) \cap A, k = 1, 2, 3\dots$

  \hspace{1cm}    \textit{while} $X_k \neq X_{k-1}$
  \medskip

  onde $X_0 = p$ e $B$ é um elemento estruturante adequado.  

\clearpage
  
  & Rotulação de componentes conexos em imagens binárias: o processo de rotulação consiste em preencher com uma cor ou rótulo diferente cada um dos componentes conexos de uma imagem. O rótulo 0 é reservado para o \textit{background}. O algoritmo abaixo pode ser usado nesse processo. Recebe como entrada uma imagem binária cujos pixels têm valores no conjunto $\{0, 255\}$.

  \medskip
  \ind label = 1
  
  \ind \textit{foreach} pixel p
  
  \ind \ind \textit{if} value(p) == 255
  
  \ind \ind \ind value(p) = label

  \ind \ind \ind Q.enqueue(p)

  \ind \ind \ind \textit{while} !Q.empty()

  \ind \ind \ind \ind c = Q.dequeue()

  \ind \ind \ind \ind \textit{foreach} d = neighbor(c)

  \ind \ind \ind \ind \ind \textit{if} value(d) == 255

  \ind \ind \ind \ind \ind \ind value(d) = label
  
  \ind \ind \ind \ind \ind \ind Q.enqueue(d)

  \ind \ind \ind label++
  
  \medskip
  
\end{easylist}
